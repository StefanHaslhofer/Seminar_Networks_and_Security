% v2020-09-21

\documentclass[11pt,a4paper]{article} %coverpage uses 12pt
\usepackage[utf8]{inputenc}

\usepackage{fancyhdr} % Fancy headings
\usepackage{amsmath} % mathematical features 
\usepackage{multirow}
\usepackage{xspace}
\usepackage{longtable}
\usepackage{rotating}
\usepackage{listings} % source code printing
\usepackage{hyperref} % links in pdf
\usepackage{booktabs,colortbl} % more commands for tables and coloring
\usepackage{geometry}
\usepackage[ngerman,english]{babel}
\usepackage{graphicx} % figures
\usepackage{float} % better positioning for floats (figures, tables)
\usepackage[T1]{fontenc}
\usepackage{listings}
\usepackage{acronym}
\usepackage{xcolor}
\usepackage{caption}
\usepackage{chngcntr}

\captionsetup[lstlisting]{ format=plain, singlelinecheck=false, margin=0pt}

% Numbering for equations with section.equation number
\renewcommand{\theequation}{\arabic{section}.\arabic{equation}}
% Numbering for figures with section.figure number
\renewcommand\thefigure{\arabic{section}.\arabic{figure}}
% Numbering for tables with section.table number
\renewcommand\thetable{\arabic{section}.\arabic{table}}


% Each section starts on a new page
\let\stdsection\section 
% Each section starts on a new page
\renewcommand\section{\clearpage\newpage\stdsection}

\lstdefinestyle{javacode}{
	belowcaptionskip=1\baselineskip,
	breaklines=true,
	numbers=left,
	frame=top,frame=bottom,
	numberblanklines=false,
	breaklines=true,
	numberstyle={\scriptsize},
	stepnumber=1,
	numbersep=1pt,
	xleftmargin=\parindent,
	language=Java,
	showstringspaces=false,
	basicstyle=\footnotesize\ttfamily,
	keywordstyle=\bfseries\color{green!40!black},
	commentstyle=\itshape\color{purple!40!black},
	identifierstyle=\color{blue},
	stringstyle=\color{orange},
}
\lstset{escapechar=@,style=javacode}

% coverpage template: https://www.jku.at/studieren/studium-von-a-z/abschlussarbeiten/masterarbeit/
% master thesis document structure: http://informatik.jku.at/teaching/stuko/news/11-03-30.html


% ToDo for student:
% - Edit coversheet.tex (language, title, name, ...)
% - Write abstract (german/english) 
%   - Bachelor Thesis: in german or english depending on the language of your thesis
%   - Master Thesis: in german and english
% - Write content of the thesis
% - Add references to the reference.bib

\begin{document}
\counterwithin{lstlisting}{section}

% insert coversheet
\input{coversheet}

% insert  affidavit / eidesstattliche erklärung
\newpage
\section*{Statutory Declaration}
\fancyhead[L]{Statutory Declaration}
\addcontentsline{toc}{section}{Statutory Declaration} 

I hereby declare under oath, that the submitted thesis has been written solely by me without any third-party assistance. Only the declared sources and/or resources have been used.
Sources for all literal, paraphrased and cited quotes have been accurately credited.

The submitted document here present is identical to the electronically submitted document.

I am aware that the violation of this regulation will lead to failure of the thesis.

\vspace{2.5cm}
\parbox{4cm}{\hrule
\strut \footnotesize Name} \hfill\parbox{4cm}{\hrule
\strut \footnotesize Signature}\\[2cm]

\parbox{4cm}{\hrule
	\strut \footnotesize Place, Date}


% set up of all pages and add the first pages (abstract and TOC)
\input{setup/setup_pages}

%------------------------------------
%         Start of Content 
%------------------------------------

\section{Introduction}
\subsection{Subsection}
\subsubsection{Subsubsection}

\section{Additional Chapter}
\subsection{Additional Chapter Level 2}
\subsubsection{Additional Chapter Level 3}

\section{Introduction to \LaTeX}
Since \LaTeX\ is widely used in academia and industry, there are many free introductions to the language. There is the wiki guide at \url{https://en.wikibooks.org/wiki/LaTeX} and also a guide from the Overleaf Online-LaTeX-Editor at \url{https://de.overleaf.com/learn}. This template was created for the Overleaf Online-LaTeX-Editor.

\subsection{Basic Functionality}
In this section, some examples are given of the basic elements used in a thesis.
For most \LaTeX\ commands optional arguments are available, which can be looked up in the various documentations for the commands.

\subsubsection{Tables}
A \verb|tabular| environment is used to create tables in \LaTeX.
\begin{table}[h] % placement specifier
  \centering
  \begin{tabular}{ll}
    \toprule
    Animal Class & Species \\
    \midrule
    \multirow{4}{*}{Mammal} & Elephant\\
                            & Horse \\
                            & Whale \\
                            & Panda \\
    \midrule
    \multirow{3}{*}{Reptile} & Snake \\
                             & Turtle \\
                             & Crocodile \\
    \midrule
    Fish & Shark \\
    \midrule
    \multirow{2}{*}{Insect} & Bee\\
                            & Ant\\
    \bottomrule
  \end{tabular}
  \caption{Adapted example from the \LaTeX\ guide at \url{https://en.wikibooks.org/wiki/LaTeX/Tables}. This example uses options from the \texttt{booktabs} the \texttt{multirow} package.}
  \label{tab:intro} % \label has to be placed AFTER \caption to produce correct cross-references.
\end{table}

\subsubsection{Images}
An image is added to a document with the \verb|\includegraphics| command as shown in Figure~\ref{fig:jku_logo}.

\begin{figure}[!htb]
    \centering
    \includegraphics[width=\textwidth]{img/JKU_Logo.png}
    \caption{JKU Logo. Always add figure sources (ie JKU, 2020)}
    \label{fig:jku_logo}
\end{figure}

Tables and figures require consecutive numbers and titles (\LaTeX~does this for you). All tables and figures taken from another source have to be cited accordingly.

\subsubsection{Mathematical Expressions}
One of the biggest advantages of \LaTeX\ is the creation of complex mathematical expressions. It is possible to insert the mathematical expression inline $\sum_{k=1}^{n} {k} = \frac{n+(n+1)}{2}$ or outside of the text as \[ \sum_{k=1}^{n} {k} = \frac{n+(n+1)}{2} \] or as numbered equation with
\begin{equation}
\sum_{k=1}^{n} {k} = \frac{n+(n+1)}{2}.
\end{equation}

\section{Listings and Acronyms}
For acronyms, please use an internal declaration in an \texttt{acronym} environment at the bottom in the main tex-file. If you supply the \texttt{[nolist]} option at \verb|\usepackage[nolist]{acronym}|, it will omit the list of acronyms accordingly. In the text, consistently use the acronym with the \verb|\ac{...}| (for singular) and \verb|\acp{...}| for plural use of the acronym. For example, the first time to mention \ac{AES} will expand the acronym, as opposed to subsequent use of \ac{AES} that does not expand it.

Listings should go included with syntax highlighting, such as in Listing \ref{src:hello-world}.

% note that the caption and label are provided directly to \lstinputlisting !
\lstinputlisting[caption=Example code,label=src:hello-world]{hello_world.java}


\subsection{References}
The references are an important part for any academic/research writing. \LaTeX\ supports different types of bibliographies to insert the references. This template uses the \textbf{BibTeX} system.
The \verb|\cite| command, makes it possible to reference entries in a \verb|.bib| file out of the text stream, e.g., as~\cite{2019_bioinformatics_ordino}.
It is also possible to add citations in captions of figures, tables, and equations, see Figure~\ref{fig:munzner_channels}
\begin{figure}[!htb]
    \centering
    \includegraphics[width=0.5\textwidth]{img/Munzner_Channels.png}
    \caption{The effectiveness of channels that modify the appearance of marks~\cite{munzner_visualization_2014}}
    \label{fig:munzner_channels}
\end{figure}

%------------------------------------
%         Acronyms 
%------------------------------------
\begin{acronym}
	\acro{AES}{Advanced Encryption Standard}%
	\acro{IDS}{Intrusion Detection System}%
	%replace the list with your relevant acronyms
\end{acronym}

%------------------------------------
%         End of Content 
%------------------------------------
% insert references
% add the references to the references.bib file
\input{setup/references}




\end{document}
